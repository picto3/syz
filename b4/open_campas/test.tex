\documentclass{jarticle}

\usepackage{amsmath, amssymb}
\usepackage{bm}
\usepackage{ascmac}
\usepackage{here}

\makeatletter

\def\@thesis{オープンキャンパス準備}
\def\id#1{\def\@id{#1}}
\def\department#1{\def\@department{#1}}

\def\@maketitle{
\begin{center}
{\huge \@thesis \par}
\vspace{10mm}
{\LARGE\bf \@title \par}
\vspace{10mm}
{\Large \@date\par}
\vspace{20mm}
{\Large \@department \par}
{\Large 学籍番号 \@id \par}
\vspace{10mm}
{\large \@author}
\end{center}
\par\vskip 1.5em
}

\makeatother

\title{TeXの練習}
\date{\today}
\department{神戸大学 理学部}
\id{1523085S}
\author{吉田 真人}


\begin{document}

\maketitle

\thispagestyle{empty}
\clearpage
\addtocounter{page}{-1}

\section{よく使いそう}
\subsection{箇条書き}
\noindent
\textbackslash begin\{itemize\} \\
  \textbackslash item いち\\
  \textbackslash item に\\
  \textbackslash item さん\\
  \textbackslash end\{itemize\}
  で次のようになる.
  \begin{itemize}
   \item いち
   \item に
   \item さん
  \end{itemize}

  \noindent
  番号付けはenumrateもしくは、itemizeで\textbackslash item[1.]のようにする.
  \begin{itemize}
   \item[1.] いち
   \item[2.] に
   \item[3.] さん
  \end{itemize}

  \subsection{表}
  \begin{table}[H]
   \begin{center}
    \caption{test}
    \begin{tabular}{|c||c|c|}
     \hline
     & 1 & 2 \\ \hline \hline
     1 & & \\ \hline
     2 & & \\ \hline
    \end{tabular}
   \end{center}
  \end{table}
   
  \subsection{数式}
   \subsubsection{\$ \$}
   文章中にそのまま出力. \\
   $a^n + b^n = c^n$みたいな感じ. \\
   分数などは$\frac{a}{b}$みたいに小さくなってしまうので
   \$\textbackslash displaystyle 数式 \$とすると
   $\displaystyle \frac{a}{b}$のように少しよくなる.
   
   \subsubsection{\$\$ \$\$}
   新しい行で真ん中に出力.
   $$\sum_{i=1}^{n} i = \frac{n(n+1)}{2}$$
   \textbackslash begin\{equation\}でもOKだが式番号が入る.
   \begin{equation}
    \sum_{i=1}^{n} i = \frac{n(n+1)}{2}
   \end{equation}
   
   \subsubsection{複数行の数式}
   eqnarrayを使う.
   \begin{eqnarray*}
    \sin2x & = & \sin(x + x) \\
    & = & \sin x \cos x + \cos x \sin x \\
    & = & 2\sin x \cos x
   \end{eqnarray*}
   
   
   
   $\displaystyle \int_{1}^{a} \frac{1}{x} dx$
\end{document}